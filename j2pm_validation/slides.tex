% ============================================================
%  J2 — APRÈS-MIDI : Sklearn, Généralisation & Overfitting
%  Julien Rolland — M2 Développement Fullstack
% ============================================================
\documentclass[aspectratio=169, 10pt]{beamer}
% ============================================================
%  PREAMBLE COMMUN — IA, Deep Learning & Machine Learning
%  Julien Rolland — M2 Développement Fullstack
% ============================================================

% --- Langue & encodage ---
\usepackage[utf8]{inputenc}
\usepackage[T1]{fontenc}
\usepackage{babel}
\babelprovide[import, main]{french}

% --- Thème Beamer ---
\usetheme{Madrid}
\usecolortheme{default}

% Palette de bleu académique
\definecolor{jedy_blue}{RGB}{0, 51, 102}       % bleu foncé principal
\definecolor{jedy_mid}{RGB}{0, 102, 179}        % bleu moyen accent
\definecolor{jedy_light}{RGB}{204, 221, 240}    % bleu très clair (fond boxes)
\definecolor{jedy_alert}{RGB}{180, 30, 30}      % rouge pour alertes
\definecolor{jedy_example}{RGB}{0, 120, 60}     % vert pour exemples

% Application des couleurs sur le thème Madrid
\setbeamercolor{palette primary}{bg=jedy_blue, fg=white}
\setbeamercolor{palette secondary}{bg=jedy_mid, fg=white}
\setbeamercolor{palette tertiary}{bg=jedy_blue, fg=white}
\setbeamercolor{palette quaternary}{bg=jedy_blue, fg=white}
\setbeamercolor{structure}{fg=jedy_blue}
\setbeamercolor{frametitle}{bg=jedy_blue, fg=white}
\setbeamercolor{title}{bg=jedy_blue, fg=white}
\setbeamercolor{block title}{bg=jedy_mid, fg=white}
\setbeamercolor{block body}{bg=jedy_light, fg=black}
\setbeamercolor{block title alerted}{bg=jedy_alert, fg=white}
\setbeamercolor{block body alerted}{bg=jedy_light, fg=black}
\setbeamercolor{block title example}{bg=jedy_example, fg=white}
\setbeamercolor{block body example}{bg=jedy_light, fg=black}

% --- Typographie ---
\usepackage{lmodern}
\setbeamerfont{title}{size=\Large, series=\bfseries}
\setbeamerfont{frametitle}{size=\normalsize, series=\bfseries}

% --- Navigation : suppression des icônes de navigation par défaut ---
\setbeamertemplate{navigation symbols}{}

% --- Numérotation des slides ---
\setbeamertemplate{footline}{%
  \leavevmode%
  \hbox{%
    \begin{beamercolorbox}[wd=.333\paperwidth,ht=2.25ex,dp=1ex,center]{author in head/foot}%
      \usebeamerfont{author in head/foot}\insertshortauthor
    \end{beamercolorbox}%
    \begin{beamercolorbox}[wd=.334\paperwidth,ht=2.25ex,dp=1ex,center]{title in head/foot}%
      \usebeamerfont{title in head/foot}\insertshorttitle
    \end{beamercolorbox}%
    \begin{beamercolorbox}[wd=.333\paperwidth,ht=2.25ex,dp=1ex,right]{date in head/foot}%
      \usebeamerfont{date in head/foot}
      \insertframenumber{} / \inserttotalframenumber\hspace*{2ex}
    \end{beamercolorbox}%
  }%
  \vskip0pt%
}

% --- Maths ---
\usepackage{amsmath, amssymb, amsthm}
\usepackage{bm}          % vecteurs en gras : \bm{w}

% --- Code source ---
\usepackage{listings}
\usepackage{xcolor}

\lstdefinestyle{pythonstyle}{
  language=Python,
  basicstyle=\ttfamily\footnotesize,
  keywordstyle=\color{jedy_blue}\bfseries,
  commentstyle=\color{gray}\itshape,
  stringstyle=\color{jedy_example},
  numberstyle=\tiny\color{gray},
  numbers=left,
  numbersep=5pt,
  frame=single,
  framerule=0.4pt,
  rulecolor=\color{jedy_light},
  backgroundcolor=\color{jedy_light!40},
  breaklines=true,
  showstringspaces=false,
  tabsize=4,
}
\lstset{style=pythonstyle}

% Alias pratique pour code inline
\newcommand{\code}[1]{\texttt{\small#1}}

% --- Graphiques ---
\usepackage{graphicx}
\usepackage{tikz}
\usetikzlibrary{arrows.meta, positioning, shapes.geometric, fit, calc}
\usepackage{pgfplots}
\pgfplotsset{compat=1.18}

% --- Tableaux ---
\usepackage{booktabs}
\usepackage{array}

% --- Icônes (optionnel, nécessite fontawesome5) ---
% \usepackage{fontawesome5}

% --- Macros ML/DL courantes ---
\newcommand{\R}{\mathbb{R}}
\newcommand{\E}{\mathbb{E}}
\newcommand{\Loss}{\mathcal{L}}
\newcommand{\dataset}{\mathcal{D}}
\newcommand{\X}{\mathbf{X}}
\newcommand{\y}{\mathbf{y}}
\newcommand{\w}{\mathbf{w}}
\newcommand{\W}{\mathbf{W}}
\newcommand{\grad}{\nabla}
\newcommand{\T}{^{\top}}         % transposée : \X\T
\newcommand{\lr}{\alpha}         % learning rate
\newcommand{\norm}[1]{\left\|#1\right\|}

% Encadré "Objectif pédagogique" en début de section
\newenvironment{objectif}{%
  \begin{alertblock}{Objectif}%
}{%
  \end{alertblock}%
}

% --- Infos du cours (remplacer dans chaque slides.tex) ---
\author[J. Rolland]{Julien Rolland}
\institute[Jedy]{Formation M2 Développement Fullstack}


\title[Généralisation \& Overfitting]{Sklearn, Généralisation \& Overfitting}
\subtitle{Jour 2 --- Après-midi}
\date{Jour 2}

% ============================================================
\begin{document}
% ============================================================

\begin{frame}
  \titlepage
\end{frame}

\begin{frame}{Plan du module}
  \tableofcontents
\end{frame}

% ============================================================
\section{Le Vrai Objectif : Généraliser}
% ============================================================

\begin{frame}{Le Vrai Objectif : Généraliser}
  \begin{columns}[T]
    \begin{column}{0.48\textwidth}
      \begin{alertblock}{Le Piège}
        \begin{itemize}
          \item Minimiser $J(\Theta, X_{\text{train}})$ = problème résolu
          \item La performance sur le train \textbf{ne compte pas}
        \end{itemize}
      \end{alertblock}

      \smallskip
      \begin{block}{Ce qu'on veut vraiment}
        \begin{itemize}
          \item Données \textbf{non-vues} (unseen data)
          \item Capter la loi sous-jacente
          \item Ignorer le bruit du dataset
        \end{itemize}
      \end{block}
    \end{column}
    \begin{column}{0.48\textwidth}
      \begin{exampleblock}{Définition}
        \textbf{Généralisation :} capacité à performer
        sur des exemples jamais vus pendant l'entraînement.
      \end{exampleblock}

      \smallskip
      % TODO : courbe train loss vs val loss (epochs)
    \end{column}
  \end{columns}
\end{frame}

% ============================================================
\section{L'Ennemi N°1 : l'Overfitting}
% ============================================================

\begin{frame}{Underfitting vs Overfitting}
  \begin{columns}[T]
    \begin{column}{0.48\textwidth}
      \begin{block}{Underfitting}
        \begin{itemize}
          \item Modèle trop \textbf{simple}
          \item Passe à côté de la tendance
          \item Biais élevé
        \end{itemize}
      \end{block}

      \smallskip
      \begin{alertblock}{Overfitting}
        \begin{itemize}
          \item Modèle trop \textbf{complexe}
          \item Relie les points, délire entre eux
          \item Variance élevée
        \end{itemize}
      \end{alertblock}
    \end{column}
    \begin{column}{0.48\textwidth}
      % TODO : courbe classique underfitting / bon fit / overfitting
      % (polynôme de degré 1, 3, 15 sur données bruitées)
    \end{column}
  \end{columns}
\end{frame}

% ---

\begin{frame}{Mémorisation vs Généralisation}
  \begin{columns}[T]
    \begin{column}{0.48\textwidth}
      \begin{block}{Théorème de Cover}
        \begin{itemize}
          \item Dans $\mathbb{R}^D$, $N \leq D$ points
                sont toujours linéairement séparables
          \item Un modèle assez complexe peut
                \textbf{mémoriser} n'importe quel dataset
          \item Accuracy train $= 100\%$ ne signifie rien
        \end{itemize}
      \end{block}
    \end{column}
    \begin{column}{0.48\textwidth}
      \begin{exampleblock}{Analogie MNIST}
        \begin{itemize}
          \item Mémoriser les pixels exacts de chaque «3»
          \item Incapable de reconnaître un «3» écrit
                avec un stylo différent
          \item $\Rightarrow$ 0\% de généralisation
        \end{itemize}
      \end{exampleblock}

      \smallskip
      % TODO : exemple visuel 2 "3" différents
    \end{column}
  \end{columns}
\end{frame}

% ============================================================
\section{Biais-Variance}
% ============================================================

\begin{frame}{Le Compromis Biais-Variance}
  \begin{columns}[T]
    \begin{column}{0.48\textwidth}
      \begin{block}{Biais élevé --- Underfitting}
        \begin{itemize}
          \item Erreur systématique
          \item Modèle trop rigide (trop peu de paramètres)
          \item Train error $\approx$ Val error, les deux élevées
        \end{itemize}
      \end{block}

      \smallskip
      \begin{alertblock}{Variance élevée --- Overfitting}
        \begin{itemize}
          \item Sensible aux variations du train
          \item Train error $\ll$ Val error
          \item Modèle trop riche (trop de paramètres)
        \end{itemize}
      \end{alertblock}
    \end{column}
    \begin{column}{0.48\textwidth}
      % TODO : courbe biais-variance vs complexité (U-shape)

      \smallskip
      \begin{exampleblock}{Levier : les Hyperparamètres}
        \begin{itemize}
          \item Profondeur d'arbre, nombre de neurones
          \item Coefficient de régularisation $\lambda$
          \item Taille du dataset
        \end{itemize}
      \end{exampleblock}
    \end{column}
  \end{columns}
\end{frame}

% ============================================================
\section{Méthodologie : Train / Validation / Test}
% ============================================================

\begin{frame}{Le Split Standard}
  \begin{columns}[T]
    \begin{column}{0.52\textwidth}
      \begin{block}{3 ensembles distincts}
        \begin{enumerate}
          \item \textbf{Train set} --- ajuster $W$, $b$
          \item \textbf{Validation set} --- choisir les hyperparamètres
          \item \textbf{Test set} --- coffre-fort,
                ouvert \textbf{une seule fois} à la fin
        \end{enumerate}
      \end{block}

      \smallskip
      \begin{alertblock}{Règle d'or : Data Leakage}
        Ne \textbf{jamais} entraîner sur le test set.\\
        Ne \textbf{jamais} utiliser le test pour choisir les hyperparamètres.\\
        Le test set doit rester \textbf{invisible}.
      \end{alertblock}
    \end{column}
    \begin{column}{0.44\textwidth}
      % TODO : schéma train/val/test (barres horizontales proportionnelles)
    \end{column}
  \end{columns}
\end{frame}

% ============================================================
\section{Cross-Validation}
% ============================================================

\begin{frame}{K-Fold Cross-Validation}
  \begin{columns}[T]
    \begin{column}{0.50\textwidth}
      \begin{alertblock}{Le Problème}
        \begin{itemize}
          \item Peu de données $\Rightarrow$ split instable
          \item Un seul val set $\Rightarrow$ variance élevée de l'estimation
        \end{itemize}
      \end{alertblock}

      \smallskip
      \begin{block}{Solution : K-Fold}
        \begin{itemize}
          \item Découper le train en $K$ morceaux (\textit{folds})
          \item $K$ rotations : chaque fold sert de validation
          \item Score final = moyenne sur $K$ runs
          \item Utilisation maximale des données disponibles
        \end{itemize}
      \end{block}
    \end{column}
    \begin{column}{0.46\textwidth}
      % TODO : schéma K-Fold (grille K×N avec fold de validation coloré)
    \end{column}
  \end{columns}
\end{frame}

% ============================================================
\section{Scikit-Learn}
% ============================================================

\begin{frame}{L'API Estimator de sklearn}
  \begin{columns}[T]
    \begin{column}{0.48\textwidth}
      \begin{block}{Interface universelle}
        \begin{itemize}
          \item \texttt{.fit(X, y)} --- entraînement
          \item \texttt{.predict(X)} --- inférence
          \item \texttt{.score(X, y)} --- évaluation
        \end{itemize}
      \end{block}

      \smallskip
      \begin{exampleblock}{Modèles clés}
        \begin{itemize}
          \item \texttt{LogisticRegression} (softmax)
          \item \texttt{StandardScaler}
          \item \texttt{cross\_val\_score}
        \end{itemize}
      \end{exampleblock}
    \end{column}
    \begin{column}{0.48\textwidth}
      \begin{block}{Pipeline}
        \begin{itemize}
          \item Enchaîner Scaler $\to$ Modèle
          \item Évite le Data Leakage sur le scaling
          \item \texttt{Pipeline([('scaler', StandardScaler()),}
                \texttt{\phantom{Pipeline(}'('clf', LogisticRegression())])}
        \end{itemize}
      \end{block}

      \smallskip
      % TODO : schéma pipeline (blocs enchaînés)
    \end{column}
  \end{columns}
\end{frame}

% ============================================================
\section{TP : MNIST avec sklearn}
% ============================================================

\begin{frame}{TP --- Setup \& Baseline}
  \begin{columns}[T]
    \begin{column}{0.48\textwidth}
      \begin{block}{Dataset MNIST}
        \begin{itemize}
          \item 70 000 images $28 \times 28$ ($= 784$ features)
          \item 10 classes (chiffres 0--9)
          \item Split standard : 60 000 train / 10 000 test
        \end{itemize}
      \end{block}

      \smallskip
      \begin{block}{Baseline}
        \begin{itemize}
          \item \texttt{LogisticRegression} (Softmax)
          \item Normalisation \texttt{StandardScaler}
          \item Métriques : accuracy, matrice de confusion
        \end{itemize}
      \end{block}
    \end{column}
    \begin{column}{0.48\textwidth}
      % TODO : grille d'exemples MNIST (chiffres 0-9)
    \end{column}
  \end{columns}
\end{frame}

% ---

\begin{frame}{TP --- Manipuler l'Overfitting}
  \begin{columns}[T]
    \begin{column}{0.52\textwidth}
      \begin{alertblock}{Observer l'Overfitting}
        \begin{itemize}
          \item Entraîner sur $N = 500$ images
          \item Train accuracy $\approx 100\%$,
                Test accuracy $\approx 60\%$
          \item Le modèle mémorise
        \end{itemize}
      \end{alertblock}

      \smallskip
      \begin{exampleblock}{Remèdes}
        \begin{itemize}
          \item \textbf{Plus de données} --- augmenter $N$
          \item \textbf{Régularisation L2} (\texttt{C} dans sklearn)
                --- pénalise les grands poids
          \item \textbf{Régularisation L1} --- sparse weights
        \end{itemize}
      \end{exampleblock}
    \end{column}
    \begin{column}{0.44\textwidth}
      % TODO : courbe train acc / test acc en fonction de N
    \end{column}
  \end{columns}
\end{frame}

% ---

\begin{frame}{TP --- Analyse des Erreurs}
  \begin{columns}[T]
    \begin{column}{0.44\textwidth}
      \begin{block}{Matrice de Confusion}
        \begin{itemize}
          \item Quels chiffres sont confondus ?
          \item Ex. : 4 vs 9, 3 vs 8, 5 vs 6
          \item Orienter les efforts d'amélioration
        \end{itemize}
      \end{block}

      \smallskip
      \begin{exampleblock}{Aller plus loin}
        \begin{itemize}
          \item \texttt{cross\_val\_score} pour estimer la variance
          \item Grid search sur $C$ (régularisation)
        \end{itemize}
      \end{exampleblock}
    \end{column}
    \begin{column}{0.52\textwidth}
      % TODO : matrice de confusion MNIST (10×10)
    \end{column}
  \end{columns}
\end{frame}

% ============================================================
\end{document}
